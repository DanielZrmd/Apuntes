\documentclass{article}
\usepackage{amsmath, amsthm, amssymb, amsfonts}
\usepackage{thmtools}
\usepackage{tikz}
\usepackage[margin=4cm]{geometry}
\usepackage{graphicx}
\usepackage{setspace}
\usepackage{geometry}
\usepackage{float}
\usepackage{hyperref}
\usepackage{enumitem}
\usepackage[utf8]{inputenc}
\usepackage[english]{babel}
\usepackage{framed}
\usepackage[dvipsnames]{xcolor}
\usepackage{tcolorbox}
\colorlet{LightGray}{White!90!Periwinkle}
\colorlet{LightOrange}{Orange!15}
\colorlet{LightGreen}{Green!15}
\usepackage{hyperref}  

\newcommand{\HRule}[1]{\rule{\linewidth}{#1}}

\declaretheoremstyle[name=Theorem,]{thmsty}
\declaretheorem[style=thmsty,numberwithin=section]{theorem}
\tcolorboxenvironment{theorem}{colback=LightGray}

\declaretheoremstyle[name=Proposition,]{prosty}
\declaretheorem[style=prosty,numberlike=theorem]{proposition}
\tcolorboxenvironment{proposition}{colback=LightOrange}

\declaretheoremstyle[name=Principle,]{prcpsty}
\declaretheorem[style=prcpsty,numberlike=theorem]{principle}
\tcolorboxenvironment{principle}{colback=LightGreen}

\setstretch{1.2}
\geometry{
    textheight=9in,
    textwidth=5.5in,
    top=1in,
    headheight=12pt,
    headsep=25pt,
    footskip=30pt
}

% ------------------------------------------------------------------------------

\begin{document}

% ------------------------------------------------------------------------------
% Cover Page and ToC
% ------------------------------------------------------------------------------

\title{ \normalsize \textsc{}
		\\ [2.0cm]
		\HRule{1.5pt} \\
		\LARGE \textbf{\uppercase{Apuntes Álgebra I}
		\HRule{2.0pt} \\ [0.6cm] \LARGE{} \vspace*{10\baselineskip}}
		}
\date{}
\author{\textbf{Autor} \\ 
		Daniel Zermeño \\
		ESFM \\
		2025}

\maketitle
\newpage

\tableofcontents
\newpage

%---------------------------------------------------------------------------------------------------------------

\section*{Introducción}
\addcontentsline{toc}{section}{Introducción}
    Los números Naturales son la base de todos los demás y de aquí se construirán más números.


    Este conjunto se denota por $\mathbb{N}$ y sus elementos son:
    \[
    \mathbb{N} := \{1, 2, 3, \dots\}
    \]

    Una manera de construir los axiomas de Peano es la siguiente:

    \begin{enumerate}[label=\roman*)]
    \item $1 \in \mathbb{N}$
    
    \item Si $n \in \mathbb{N}$, entonces su sucesor $n + 1$ también pertenece a $\mathbb{N}$
    
    \item Si $n + 1 = m + 1$, entonces $n = m$
    
    \item $1 \neq n + 1$ para todo $n \in \mathbb{N}$
    
    \item Principio de inducción:  
    Sea \(E \subseteq \mathbb{N}\) tal que
    \begin{itemize}
    \item[\(\cdot)\)] \(1 \in E.\)
    \item[\(\cdot\ \cdot)\)] Si \(n \in E\), entonces \(n + 1 \in E.\)
    \end{itemize}

    Entonces $\mathbb{N} = E$.
    \end{enumerate}


%---------------------------------------------------------------------------------------------------------------
\newpage
\section{Propiedades de los números enteros}
 
    Considere el siguiente conjunto denotado por $\mathbb{Z}$ :
    $$
    \mathbb{Z}=\mathbb{N} \cup\{0\} \cup\{\mathbb{N}\}=\{-\ldots .,-3,-2,-1,0,1,2,3, \ldots .+\}
    $$

     Definimos las operaciones de suma y producto, denotadas por \(+\) y \(\cdot\).  El conjunto \(\mathbb{Z}\), junto con estas dos operaciones, satisface las siguientes propiedades:

    Sean $a, b, c \in \mathbb{Z}$
    \begin{enumerate}[label=\roman*)]
    \item $(a+b)+c=a+(b+c)$
    
    \item $a+b=b+a$
    
    \item $\exists \quad 0 \in \mathbb{Z}$ tal que $a+0=a$
    
    \item $\exists \bar{a} \in \mathbb{Z}$ tal que $a+\bar{a}=0$
    \end{enumerate}
        
    Note que la de la propiedad \(\text{iv}\) es \(-a\).
\newpage

    \subsection*{Definición 1.1 (Divisibilidad en \(\mathbb{Z}\))}
    \addcontentsline{toc}{subsection}{\texorpdfstring{Definición 1.1 (Divisibilidad en \(\mathbb{Z}\))}{Definición 1.1 (Divisibilidad en Z)}}


        Sean $a, b \in \mathbb{Z}$. 
        
        Se dice que a divide a $b$ o que es  divisible por $a$ si existe $c \in \mathbb{Z}$ tal que $b=a\cdot  c$

        Si $a$ divide a $b$, esto se denota como $a\mid b$ y si $a$ no divide a $b$ se denota como $a\nmid b$.
        
        
        Ejemplos:
        \begin{enumerate}[label=\roman*)]
            \item $5\mid 50$ ya que $\exists x$ tal que $5\cdot 10=50$, $x=10$
            \item $5\nmid 16$ ya que $\nexists x$ tal que $5\cdot (x)=16$
        \end{enumerate}

    \subsection*{Teorema 1.2}
        Propiedades de la divisibilidad: Sean $a.b.c\in \mathbb{Z}$. Entonces.
            \begin{enumerate}[label=\roman*)]
                \item $1\mid a$ y $-1\mid a$
                
            \end{enumerate}

        

%---------------------------------------------------------------------------------------------------------------
\newpage
\section{Complejos}

    Un número complejo es una expresión de la forma $a+ib$, donde $a$ y $b$ son números reales e $i$ es un símbolo
    Al conjunto de los números complejos se le denota como:
    $$
    \mathbb{C}:=\{a+i b \mid a, b \in \mathbb{R}\}
    $$

    \begin{enumerate}[label=\roman*)]
    \item Parte real de un número complejo.
    
    En la expresión $a+i b$, al número real $a$ se le conoce como la parte real del complejo $a+i b$ y se le denota por:
    $$
    \operatorname{Re}(a+i b)=a .
    $$

    
    \item Parte imaginaria de un número complejo.
    
    En la expresión $a+i b$, $a l$ número real b se le conoce como parte imaginaria del complejo $a+i b$ y se le denota por:
    $$
    \operatorname{Im}(a+i b)=b
    $$

    \item Forma normal de un número complejo.
    
    La expresión $a+i b$ se conoce como forma normal de un número complejo, otra manera de denotar a los números complejos es como una  pareja ordenada de dos números
    $$
    a+ib =(a, b)
    $$

    \item Plano Complejo:
     
    \begin{center}
    \begin{tikzpicture}[scale=1.2]
    % Ejes
    \draw[->] (-0.5,0) -- (4.5,0) node[right] {Re};
    \draw[->] (0,-0.5) -- (0,4) node[above] {Im};
 

    % Punto (a,b) - Coordinates are chosen to match the proportions in the image
    % Let 'a' be the real part (x-coordinate) and 'b' be the imaginary part (y-coordinate)
    \def\realpart{3} % x-coordinate
    \def\imagpart{2} % y-coordinate
 

    \coordinate (O) at (0,0);
    \coordinate (P_real) at (\realpart,0);
    \coordinate (P_imag) at (0,\imagpart);
    \coordinate (P_complex) at (\realpart,\imagpart);
 

    % Vector a + ib
    \draw[thick, blue, ->] (O) -- (P_complex) node[midway, sloped, above] {$a + ib$};
 

    % Líneas punteadas (catetos)
    \draw[dashed] (P_real) -- (P_complex); % Vertical dashed line
    \draw[dashed] (P_imag) -- (P_complex); % Horizontal dashed line (implied from origin to y-axis)
 

    % Etiquetas de catetos - Correctly placed
    \node[below] at (\realpart/2,0) {$a$}; % Label 'a' for the real component
    \node[right] at (\realpart,\imagpart/2) {$b$}; % Label 'b' for the imaginary component
 

    % Ángulo theta
    % Calculate angle dynamically: atan2(y,x) for the angle from positive x-axis
    \pgfmathsetmacro{\angleval}{atan2(\imagpart,\realpart)}
    \draw[->, thick] (1,0) arc[start angle=0,end angle=\angleval,radius=1cm];
    \node at (1.3,0.3) {$\theta$};
 

    % Equivalencia
    \node[above right] at (P_complex) {$a+ib \approx \mathbb{R}^2$};
    \end{tikzpicture}
    \end{center}
    \end{enumerate}
    \newpage

    
\end{document}
